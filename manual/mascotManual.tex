\documentclass[a4paper]{article}

\usepackage{amsmath}
\usepackage{amssymb}
\usepackage{enumerate}
\usepackage{paralist}
\usepackage{xspace}
\usepackage{fullpage}
\usepackage{color}

\newcommand{\mascot}{\textsf{MASCOT}\xspace}
\newcommand{\scots}{\textsf{SCOTS}\xspace}

\begin{document}
%	\section{Scope of the Repeatability Evaluation}
%		The following elements of the paper can be evaluated using the provided REP package:
%		\begin{compactitem}
%			\item Table~1
%			\item Table~2
%			\item Table~3
%			\item Figure~4
%			\item Figure~5
%		\end{compactitem}
	\section{System Requirements}
		We have built \mascot on top of \scots \cite{Scots}. Therefore, the system requirements and the installation procedure for our software are the same as for \scots. As for \scots, you need a working C++ developer environment, the CUDD library \cite{somenzi2009cudd}, and a working installation of a recent version of MATLAB (the latter is only used for the visualization of some results). Following are the requirements as detailed in the manual of \scots:
		\begin{enumerate}
			\item A working C++ developer environment:
				\begin{itemize}
					\item Mac OS: Install the latest version of Xcode.app including the command line tools.
					\item Linux: Most Linux OS distributions already include all the necessary tools already. We used the compiler GCC 4.9.x.
					\item Windows: \scots has not been tested on a Windows machine.
				\end{itemize}
			\item A working installation of the CUDD library with
				\begin{itemize}
					\item the C++ object oriented wrapper,
					\item the dddmp library, and
					\item the shared library
				\end{itemize}
				option enabled. The package follows the usual configure, make and make install installation routine. We use cudd-3.0.0, with the configuration
				\begin{verbatim}
					$ ./configure --enable-shared --enable-obj --enable-dddmp --prefix=/opt/local/
				\end{verbatim}
				The BDD installation can be tested by compiling a dummy program, e.g. test.cc
				%\hline
				\begin{verbatim}
					#include<iostream>
					#include "cuddObj.hh"
					#include "dddmp.h"
					int main () {
						Cudd mgr(0,0);
						BDD x = mgr.bddVar();
					}
				\end{verbatim}
				%\hline
				by running
				\begin{verbatim}
					$ g++ test.cc -I/opt/local/include -L/opt/local/lib -lcudd
				\end{verbatim}
				It has been reported that on some linux machines, the header files \texttt{util.h} and \texttt{config.h} were missing in \texttt{/opt/local}, and the fix is to manually copy the files to \texttt{/opt/local/include} by running
				\begin{verbatim}
					 $ sudo cp ./util/util.h /opt/local/include/
					 $ sudo cp ./config.h /opt/local/include/
				\end{verbatim}
			\item A working installation of a recent version of MATLAB. We conducted our experiments with version R2016a (9.0.0.341360) 64-bit. To compile the mex files:
				\begin{enumerate}[(a)]
					\item open MATLAB and setup the mex compiler by
						\begin{verbatim}
							>> mex -setup C++
						\end{verbatim}
					\item Open a terminal and go to \texttt{./mfiles/mexfiles/} (within the \mascot directory, see below). Edit the \texttt{Makefile} and adjust the variables
						\begin{verbatim}
							MATLABROOT and CUDDPATH
						\end{verbatim}
						and run 
						\begin{verbatim}
						$ make
						\end{verbatim}
						(The MATLAB root directory is returned when typing \texttt{matlabroot} in the command window of MATLAB.)
				\end{enumerate}
		\end{enumerate}
	\section{Directory Structure}
	After you unzip \mascot, you will see the following directory structure:
		\begin{itemize}
			\item \texttt{./bdd/} Contains the C++ source code for the \scots and \mascot classes which use Binary Decision Diagrams as the underlying data structure
			\item \texttt{./doxygen/} C++ Class documentation directory
			\item \texttt{./manual/} Both \scots and \mascot manuals
			\item \texttt{./mascot\_examples/} Examples which were used in the papers \cite{hsu2018lazy1, hsu2018lazy2} demonstrating the usage and performance of \mascot 
			\item \texttt{./mfiles} Contains an mfile as a wrapper to the mex-file functions
			\item \texttt{./mfiles/mexfiles/} mex-file to read the C++ output from file
			\item \texttt{./scots\_examples/} Some C++/Matlab programs demonstrating the usage of basic \scots
			\item \texttt{./test/} Some C++/Matlab test programs
			\item \texttt{./utils/} Some C++ header files used by the source codes in \texttt{./bdd/}
		\end{itemize}
	\section{Getting Started}
	 	For validating any of the tables/figures presented in the paper
	 		\begin{enumerate}
%	 			\item go to the corresponding sub-directory in \texttt{./REP\_examples/}. For example, for validating the data in Table 1 of the paper, go to \texttt{./REP\_examples/Table1}.
%	 			\item Inside you will find different sub-directories containing different test cases as presented in the corresponding table/figure. The names of the sub-directories are self-explanatory. For example, inside \texttt{./REP\_examples/Table1}, you will find \texttt{./REP\_examples/Table1/unicycle1A}, \texttt{./REP\_examples/Table1/unicycle2A} and \texttt{./REP\_examples/Table1/unicycle3A}, representing the 1-layered (normal \scots), 2-layered and 3-layered reachability algorithms respectively.
				\item Go to a sub-directory in one of the example folders.
	 			\item Edit the \texttt{Makefile}:
	 				\begin{enumerate}[(a)]
	 					\item adjust the compiler, and
	 					\item adjust the directories of the CUDD library.
	 				\end{enumerate}
	 			\item Compile and run the executable, for example in \texttt{./mascot\_examples/safety/dcdc/LazySafe/dcdc3A/} run
	 				\begin{verbatim}
	 					$ make
	 					$ ./dcdc
	 				\end{verbatim}
	 			%\item Use the .m files to plot results. \textcolor{red}{todo: discuss different modes}
	 		\end{enumerate}
%	 	For Figure 4, you need to perform one more step to plot the trajectory of the unicycle:
%	 	\begin{enumerate}
%	 		\item Open \texttt{MATLAB} and change the working directory to\linebreak \texttt{./REP\_examples/Figure4/unicycle3A}.
%	 		\item In the command window, run
%	 			\begin{verbatim}
%	 				>> unicyclePlot(`plot',3,C)
%	 			\end{verbatim}
%	 			by replacing \texttt{C} with the highest controller number saved in the folder \linebreak \texttt{./REP\_examples/Figure4/unicycle3A/C} (after the executable is run). E.g., if this folder contains files \texttt{C1.bdd} to \texttt{C15.bdd}, replace \texttt{C} by $15$.
%	 	\end{enumerate}
%	 	For Figure~5, Table~1, Table~2 and Table~3, the abstraction and the synthesis times for different number of layers can be found in the log files (\texttt{.txt}-file) stored in the corresponding folders. 
	 	
	 	\bibliographystyle{plain}
	 	\bibliography{reportbib}
\end{document}